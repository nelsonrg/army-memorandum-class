% Created 2019-02-24 Sun 16:11
% Intended LaTeX compiler: pdflatex
\documentclass{../armymemo}

\author{JOHN W. SMITH}\rank{Colonel}\branch{GS}\title{Chief of Staff}
\officesymbol{OFFICE SYMBOL}
\signaturedate{Date}
\memoline{MEMORANDUM FOR U.S. Army Command and General Staff College (ATZL), 100 Stimson Avenue, Ft Leavenworth, KS 66027-1352}
\subject{Using and Preparing a Memorandum}
\authority{AUTHORITY LINE}
\addencl{Enclosure 1}
\addencl{Enclosure 2}
\addencl{Enclosure 3}
\addencl{Enclosure 4}
\addcf{Director, Tactics Division}

\begin{document}

\begin{enumerate}
\item See paragraph 2-2 (of AR 25-50) for when to use a memorandum.
\item Single space the text and double space between paragraphs and subparagraphs. Insert two blank spaces after ending punctuation (period and question mark). Insert two blank spaces after a colon. When numbering subparagraphs, insert two blank spaces after parentheses.
\item When a memorandum has more than one paragraph, number the paragraphs consecutively. When paragraphs are subdivided, designate first subdivisions using lowercase letters of the alphabet and indent 1/4 inch as shown below.
  \begin{enumerate}
  \item When a paragraph is subdivided, it must have at least two subparagraphs.
  \item If there is a subparagraph ``a,'' there must be a subparagraph ``b.''
    \begin{enumerate}
    \item Designate second subdivisions by numbers in parentheses; for example (1), (2), and (3) and indent by 1/2 inch as shown.
    \item Do not subdivide beyond the third subdivision.
      \begin{enumerate}
      \item Do not indent any further than the second subdivision.
      \item Use (a), (b), (c), and so forth at this level.
      \end{enumerate}
    \end{enumerate}
  \end{enumerate}
\end{enumerate}

\end{document}


%%% Local Variables:
%%% mode: latex
%%% TeX-master: t
%%% End:
