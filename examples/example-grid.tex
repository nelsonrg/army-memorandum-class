\documentclass{../armymemo}
% used to show page layouts
\usepackage{layouts}
% In order to show baselines:
% https://tex.stackexchange.com/q/159010
\usepackage{atbegshi,picture,xcolor,hyperref}
\AtBeginShipout{%
  \AtBeginShipoutUpperLeft{%
    \color{red}%
    \put(\dimexpr 1in+\oddsidemargin,
    -\dimexpr \topmargin+\headheight+\headsep+\topskip)%
    {%
      \vtop to\dimexpr\vsize+\headheight+\headsep+\topskip+\baselineskip{
        \hrule
        \leaders\vbox to\baselineskip{\hrule width\hsize\vfill}\vfill
      }%
    }%
  }%
}

\logo{imperialseal-536x536}
\department{Galactic Fleet}
\address{Death Star}
\address{Alderaan System}
\address{Core Worlds}
\officesymbol{DS-1-OBS}
\signaturedate{10 April 2019}
\documentmark{GALACTIC CONFIDENTIAL//DEATHSTAR}

% If the subject spills into a 2nd line, it breaks \pagedesign from layout, but otherwise seems to work
\subject{On the use and testing of \LaTeX\ for document production}% and system documentation for the DS-1 Orbital Battle Station}

% \mfr

\multimemothru{U.S. Army Command and General Staff College (ATZL), 100 Stimson Avenue, Ft Leavenworth, KS 66027-1352}
\multimemothru{CHIEF OF STAFF, 501\st\ Legion, Galactic Army, Grand Army of the Republic, Galactic Empire}
\addmemoline{FOR General of the Grand Army of the Republic, Dark Lord of the Sith, Darth Vader}

% \addmemoline{MEMORANDUM THRU CHIEF OF STAFF, 501\st\ Legion, Galactic Army, Grand Army of the Republic, Galactic Empire}
% \addmemoline{FOR U.S. Army Command and General Staff College (ATZL), 100 Stimson Avenue, Ft Leavenworth, KS 66027-1352}

% \addmemoline{MEMORANDUM FOR U.S. Army Command and General Staff College (ATZL), 100 Stimson Avenue, Ft Leavenworth, KS 66027-1352}

\authority{FOR THE COMMANDER}
\author{Boba Fett}
\rank{Contractor}
\branch{Independent}
\title{Procurement of Prized Heads}

\suspensedate{1 JAN 2019}
\authority{AUTHORITY LINE}

\addencl{\LaTeX HOWTO}
\addencl{StackExchange}
\addencl{armymemo.cls}
\continuedistro
\adddistro{G.M. Tarkin}
\adddistro{Adm. Ozzel}
\adddistro{Adm. Motti}
\adddistro{V.Adm. Piett}
\addcf{Galactic J3}
\addcf{Fleet J3}

\begin{document}
\begin{enumerate}
% \item \currentlist
%   \begin{figure}
%     \listdiagram
%   \end{figure}
\item This document uses code from \url{https://tex.stackexchange.com/q/159010}
  to display the baselines in order to confirm spacing between paragraph, list
  items, and other elements of the memo.
\item This \verb!example-grid.tex! document is for testing features of the class
  and validating layout.
\item See paragraph 2-2 (of AR 25-50) for when to use a memorandum.
\item Single space the text and double space between paragraphs and
  subparagraphs. Insert two blank spaces after ending punctuation (period and
  question mark). Insert two blank spaces after a colon. When numbering
  subparagraphs, insert two blank spaces after parentheses.
\item When a memorandum has more than one paragraph, number the paragraphs
  consecutively. When paragraphs are subdivided, designate first subdivisions
  using lowercase letters of the alphabet and indent 1/4 inch as shown below.
  \begin{enumerate}
  \item When a paragraph is subdivided, it must have at least two subparagraphs.
  \item If there is a subparagraph ``a,'' there must be a subparagraph ``b.''
    \begin{enumerate}
    \item Designate second subdivisions by numbers in parentheses; for example
      (1), (2), and (3) and indent by 1/2 inch as shown.
    \item Do not subdivide beyond the third subdivision.
      \begin{enumerate}
      \item Do not indent any further than the second subdivision.
      \item Use (a), (b), (c), and so forth at this level.
      \end{enumerate}
    \end{enumerate}
  \end{enumerate}
\item Moby Dick, a classic American Novel, the text of which is public domain,
  follows in several paragraphs to make a longer example:
  \begin{enumerate}
  \item Call me Ishmael.\footnote{This is a footnote} Some years ago- never mind
    how long precisely- having little or no money in my purse\footnote{wallet},
    and nothing particular to interest me on shore, I thought I would sail about
    a little and see the watery part of the world.\footnote{aka the ocean} It is
    a way I have of driving off the spleen and regulating the
    circulation. Whenever I find myself growing grim about the mouth; whenever
    it is a damp, drizzly November in my soul; whenever I find myself
    involuntarily pausing before coffin warehouses, and bringing up the rear of
    every funeral I meet; and especially whenever my hypos get such an upper
    hand of me, that it requires a strong moral principle to prevent me from
    deliberately stepping into the street, and methodically knocking people's
    hats off- then, I account it high time to get to sea as soon as I can.  This
    is my substitute for pistol and ball. With a philosophical flourish Cato
    throws himself upon his sword; I quietly take to the ship. There is nothing
    surprising in this. If they but knew it, almost all men in their degree,
    some time or other, cherish very nearly the same feelings towards the ocean
    with me.

  \item There now is your insular city of the Manhattoes, belted round by wharves as
    Indian isles by coral reefs- commerce surrounds it with her surf. Right and
    left, the streets take you waterward. Its extreme downtown is the battery, where
    that noble mole is washed by waves, and cooled by breezes, which a few hours
    previous were out of sight of land. Look at the crowds of water-gazers there.

  \item Circumambulate the city of a dreamy Sabbath afternoon. Go from Corlears Hook to
    Coenties Slip, and from thence, by Whitehall, northward. What do you see?-
    Posted like silent sentinels all around the town, stand thousands upon thousands
    of mortal men fixed in ocean reveries. Some leaning against the spiles; some
    seated upon the pier-heads; some looking over the bulwarks of ships from China;
    some high aloft in the rigging, as if striving to get a still better seaward
    peep. But these are all landsmen; of week days pent up in lath and plaster- tied
    to counters, nailed to benches, clinched to desks. How then is this? Are the
    green fields gone? What do they here?

  \item But look! here come more crowds, pacing straight for the water, and seemingly
    bound for a dive. Strange! Nothing will content them but the extremest limit of
    the land; loitering under the shady lee of yonder warehouses will not suffice.
    No. They must get just as nigh the water as they possibly can without falling
    And there they stand- miles of them- leagues. Inlanders all, they come from
    lanes and alleys, streets avenues- north, east, south, and west. Yet here they
    all unite. Tell me, does the magnetic virtue of the needles of the compasses of
    all those ships attract them thither?

  \item Once more. Say you are in the country; in some high land of lakes. Take
    almost any path you please, and ten to one it carries you down in a dale,
    and leaves you there by a pool in the stream. There is magic in it. Let the
    most absent-minded of men be plunged in his deepest reveries—stand that man
    on his legs, set his feet a-going, and he will infallibly lead you to water,
    if water there be in all that region.  Should you ever be athirst in the
    great American desert, try this experiment, if your caravan happen to be
    supplied with a metaphysical professor. Yes, as every one knows, meditation
    and water are wedded for ever.

  \item But here is an artist. He desires to paint you the dreamiest, shadiest,
    quietest, most enchanting bit of romantic landscape in all the valley of the
    Saco. What is the chief element he employs? There stand his trees, each with
    a hollow trunk, as if a hermit and a crucifix were within; and here sleeps
    his meadow, and there sleep his cattle; and up from yonder cottage goes a
    sleepy smoke. Deep into distant woodlands winds a mazy way, reaching to
    overlapping spurs of mountains bathed in their hill-side blue. But though
    the picture lies thus tranced, and though this pine-tree shakes down its
    sighs like leaves upon this shepherd’s head, yet all were vain, unless the
    shepherd’s eye were fixed upon the magic stream before him. Go visit the
    Prairies in June, when for scores on scores of miles you wade knee-deep
    among Tiger-lilies—what is the one charm wanting?—Water—there is not a drop
    of water there! Were Niagara but a cataract of sand, would you travel your
    thousand miles to see it? Why did the poor poet of Tennessee, upon suddenly
    receiving two handfuls of silver, deliberate whether to buy him a coat,
    which he sadly needed, or invest his money in a pedestrian trip to Rockaway
    Beach? Why is almost every robust healthy boy with a robust healthy soul in
    him, at some time or other crazy to go to sea?  Why upon your first voyage
    as a passenger, did you yourself feel such a mystical vibration, when first
    told that you and your ship were now out of sight of land? Why did the old
    Persians hold the sea holy? Why did the Greeks give it a separate deity, and
    own brother of Jove? Surely all this is not without meaning. And still
    deeper the meaning of that story of Narcissus, who because he could not
    grasp the tormenting, mild image he saw in the fountain, plunged into it and
    was drowned. But that same image, we ourselves see in all rivers and
    oceans. It is the image of the ungraspable phantom of life; and this is the
    key to it all.
    
  \item Now, when I say that I am in the habit of going to sea whenever I begin
    to grow hazy about the eyes, and begin to be over conscious of my lungs, I
    do not mean to have it inferred that I ever go to sea as a passenger. For to
    go as a passenger you must needs have a purse, and a purse is but a rag
    unless you have something in it. Besides, passengers get sea-sick—grow
    quarrelsome—don’t sleep of nights—do not enjoy themselves much, as a general
    thing;—no, I never go as a passenger; nor, though I am something of a salt,
    do I ever go to sea as a Commodore, or a Captain, or a Cook. I abandon the
    glory and distinction of such offices to those who like them. For my part, I
    abominate all honorable respectable toils, trials, and tribulations of every
    kind whatsoever. It is quite as much as I can do to take care of myself,
    without taking care of ships, barques, brigs, schooners, and what not.  And
    as for going as cook,—though I confess there is considerable glory in that,
    a cook being a sort of officer on ship-board—yet, somehow, I never fancied
    broiling fowls;—though once broiled, judiciously buttered, and judgmatically
    salted and peppered, there is no one who will speak more respectfully, not
    to say reverentially, of a broiled fowl than I will. It is out of the
    idolatrous dotings of the old Egyptians upon broiled ibis and roasted river
    horse, that you see the mummies of those creatures in their huge bake-houses
    the pyramids.
  \item No, when I go to sea, I go as a simple sailor, right before the mast,
    plumb down into the forecastle, aloft there to the royal mast-head.  True,
    they rather order me about some, and make me jump from spar to spar, like a
    grasshopper in a May meadow. And at first, this sort of thing is unpleasant
    enough. It touches one’s sense of honor, particularly if you come of an old
    established family in the land, the Van Rensselaers, or Randolphs, or
    Hardicanutes. And more than all, if just previous to putting your hand into
    the tar-pot, you have been lording it as a country schoolmaster, making the
    tallest boys stand in awe of you. The transition is a keen one, I assure
    you, from a schoolmaster to a sailor, and requires a strong decoction of
    Seneca and the Stoics to enable you to grin and bear it. But even this wears
    off in time.
  \item What of it, if some old hunks of a sea-captain orders me to get a broom
    and sweep down the decks? What does that indignity amount to, weighed, I
    mean, in the scales of the New Testament? Do you think the archangel Gabriel
    thinks anything the less of me, because I promptly and respectfully obey
    that old hunks in that particular instance? Who ain’t a slave? Tell me
    that. Well, then, however the old sea-captains may order me about—however
    they may thump and punch me about, I have the satisfaction of knowing that
    it is all right; that everybody else is one way or other served in much the
    same way—either in a physical or metaphysical point of view, that is; and so
    the universal thump is passed round, and all hands should rub each other’s
    shoulder-blades, and be content.
  \item Again, I always go to sea as a sailor, because they make a point of
    paying me for my trouble, whereas they never pay passengers a single penny
    that I ever heard of. On the contrary, passengers themselves must pay. And
    there is all the difference in the world between paying and being paid. The
    act of paying is perhaps the most uncomfortable infliction that the two
    orchard thieves entailed upon us. But being paid,—-what will compare with
    it? The urbane activity with which a man receives money is really
    marvellous, considering that we so earnestly believe money to be the root of
    all earthly ills, and that on no account can a monied man enter heaven. Ah!
    how cheerfully we consign ourselves to perdition!
  \item Finally, I always go to sea as a sailor, because of the wholesome
    exercise and pure air of the fore-castle deck. For as in this world, head
    winds are far more prevalent than winds from astern (that is, if you never
    violate the Pythagorean maxim), so for the most part the Commodore on the
    quarter-deck gets his atmosphere at second hand from the sailors on the
    forecastle. He thinks he breathes it first; but not so. In much the same way
    do the commonalty lead their leaders in many other things, at the same time
    that the leaders little suspect it. But wherefore it was that after having
    repeatedly smelt the sea as a merchant sailor, I should now take it into my
    head to go on a whaling voyage; this the invisible police officer of the
    Fates, who has the constant surveillance of me, and secretly dogs me, and
    influences me in some unaccountable way—he can better answer than any one
    else. And, doubtless, my going on this whaling voyage, formed part of the
    grand programme of Providence that was drawn up a long time ago. It came in
    as a sort of brief interlude and solo between more extensive performances. I
    take it that this part of the bill must have run something like this:
    \begin{enumerate}
    \item ``\textit{Grand Contested Election for the Presidency of the United
        States.}''
    \item ``WHALING VOYAGE BY ONE ISHMAEL.''
    \item “BLOODY BATTLE IN AFGHANISTAN.”
    \end{enumerate}
  \item Though I cannot tell why it was exactly that those stage managers, the
    Fates, put me down for this shabby part of a whaling voyage, when others
    were set down for magnificent parts in high tragedies, and short and easy
    parts in genteel comedies, and jolly parts in farces—though I cannot tell
    why this was exactly; yet, now that I recall all the circumstances, I think
    I can see a little into the springs and motives which being cunningly
    presented to me under various disguises, induced me to set about performing
    the part I did, besides cajoling me into the delusion that it was a choice
    resulting from my own unbiased freewill and discriminating judgment.
  % \item Chief among these motives was the overwhelming idea of the great whale
  %   himself. Such a portentous and mysterious monster roused all my
  %   curiosity. Then the wild and distant seas where he rolled his island bulk;
  %   the undeliverable, nameless perils of the whale; these, with all the
  %   attending marvels of a thousand Patagonian sights and sounds, helped to sway
  %   me to my wish. With other men, perhaps, such things would not have been
  %   inducements; but as for me, I am tormented with an everlasting itch for
  %   things remote. I love to sail forbidden seas, and land on barbarous
  %   coasts. Not ignoring what is good, I am quick to perceive a horror, and
  %   could still be social with it—would they let me—since it is but well to be
  %   on friendly terms with all the inmates of the place one lodges in.
  % \item By reason of these things, then, the whaling voyage was welcome; the
  %   great flood-gates of the wonder-world swung open, and in the wild conceits
  %   that swayed me to my purpose, two and two there floated into my inmost soul,
  %   endless processions of the whale, and, mid most of them all, one grand
  %   hooded phantom, like a snow hill in the air.

  \end{enumerate}

\item POC for this memo is John Doe, who can be reached at \texttt{john@dev.null}
  or (555) 555-1234.
\end{enumerate}

\begin{figure}
  \centering
  \currentpage
  \drawmarginparsfalse
  \drawdimensionstrue
  \pagedesign
  \caption{Document layout}
  \label{fig:1}
\end{figure}

\end{document}